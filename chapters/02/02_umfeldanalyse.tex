\chapter{Umfeldanalysen}
\section{Umfeldanalyse von ...}
In der folgenden Umfeldanalyse werden drei ... miteinander verglichen. Anhand der Auswahlkriterien: \textit{Programmierung, Relevanz und Marktanteil} und \textit{Dokumentation und Lektüre} wird entschieden, welche ... für die Diplomarbeit am besten geeignet ist.

\subsection{Allgemeine Auswahlkriterien}
\begin{itemize}
  \item \textbf{Relevanz und Marktanteil:}\\\\ In dem Vergleichspunkt Relevanz und Marktanteil wird gezeigt, für welche Arten von Spielen die Game Engine benutzt wurde. Es soll außerdem dargestellt werden, wie stark die Game Engine auf dem Markt vertreten ist.
  \item \textbf{Programmierung:}\\\\ Die Programmierung beschreibt, wie und mit welchen Programmiersprachen in der Game Engine umgegangen wird. Es wird unterschieden zwischen dem Schreiben von Spielskripten und dem visuellen Skripten.
  \item \textbf{Dokumentation und Lektüre:}\\\\ Der Vergleichspunkt Dokumentation und Lektüre beschreibt, wie viele und welche Art von Dokumentationen verfügbar sind. 
\end{itemize}

\pagebreak

\subsection{Vergleichspunkt1}
\subsubsection{Relevanz und Marktanteil}

\subsubsection{Programmierung}

\subsubsection{Dokumentation und Lektüre (Community)}

\subsection{Entscheidung}

\pagebreak