\chapter{chapter}
\setauthorname{autor 1}

\section{section}
\subsection{subsection}
\subsubsection{subsubsection}

Unity (auch bekannt als Unity3D) ist eine Game Engine und \bettergls{ide}{1} zum kreieren von Interaktionsmedien, meißt Video Spielen. \Cite[][A history of the unity game engine]{haas2014history} \\


\bettergls{hp}{2} wird in der Fußnote genauer beschrieben und ist eingetragen in dem Glossar.


\begin{figure}[H]
    \centering
    \includegraphics[width=0.2\linewidth]{chapters/00/images/Csharp.png}
    \caption{Das Logo von der Programmiersprache C\#.}
    \label{temp1}
\end{figure}


% C#
\begin{lstlisting}[language=CSharp,caption={Hello World},label=code:hello_world]
    static void Main(string[] args)
    {
        System.Console.WriteLine("Hello World!");
    }
\end{lstlisting}  

\pagebreak
\section{seite 2 mit anderem autor}
\setauthorname{autor 2}

\begin{figure}[h]
    \centering
    \begin{minipage}[b]{0.4\textwidth}
      \centering
      \includegraphics[width=0.3\textwidth]{chapters/00/images/Csharp.png}
      \caption{Nochmal das Logo von der Programmiersprache C\#.}
      \label{temp2}
    \end{minipage}
    \hfill
    \begin{minipage}[b]{0.4\textwidth}
        \centering
        \includegraphics[width=0.3\textwidth]{chapters/00/images/latex.png}
        \caption{Hier das Latex Logo}
        \label{temp3}
    \end{minipage}
    \label{}
\end{figure}

\section{Auflistung}

\begin{itemize}
    \item item 1 
    \item item 2 
    \item item 3
\end{itemize}